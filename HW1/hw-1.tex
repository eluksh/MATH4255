% Homework 1
% Emily Lukshin
% MATH 4255 Spring 2020
% February 5 2020

% Preamble

\documentclass{report}
\usepackage[left=1in,right=1in,top=1in,bottom=1in]{geometry}
\setlength{\headheight}{23pt}
\usepackage{amsthm}
\usepackage{amsfonts}
\usepackage{amsmath}
\usepackage{amssymb}
\usepackage{mathrsfs}
\usepackage{multirow}
\usepackage{verbatim}
\usepackage{url}
\usepackage{yfonts}
\usepackage{fancyhdr}
\pagestyle{fancy}
\usepackage[colorlinks]{hyperref}
\usepackage{wrapfig}
\usepackage{graphicx}
\begin{document}


% Header

\fancyhead{}
\fancyfoot{}
\lhead{\large \textbf{Math 4255/5255}}
\chead{\Large \textbf{Homework 1}}
\rhead{\large \textbf{Emily Lukshin}}


% Custom Commands     

\newcommand{\N}{\mathbb{N}}
\newcommand{\Z}{\mathbb{Z}}
\newcommand{\Q}{\mathbb{Q}}
\newcommand{\R}{\mathbb{R}}
\newcommand{\C}{\mathbb{C}}
\newcommand{\D}{\mathbb{D}}
\newcommand{\<}{\left\langle}
\renewcommand{\>}{\right\rangle}
\renewcommand{\Re}[1]{\text{Re}\ #1}
\renewcommand{\Im}[1]{\text{Im}\ #1}
\renewcommand{\mod}[1]{(\operatorname{mod}#1)}


% Document

\noindent
{\bf Please submit your solutions (both tex and PDF files) to WyoCourses; and hand in a printed version in class. If you are a graduate student, please also finish Problem 5.}

\medskip


The following problems are mainly selected from Ross's book. 

\begin{enumerate}

% EXERCISE #1

\item Twenty workers are to be assigned to 20 different
jobs, one to each job. How many different assignments
are possible?

{\bf Solution:} 
    $$ \text{All possible assignments} =  20! \approx 2.4329 \times 10^{18} $$

% EXERCISE #2

\item For years, telephone area codes in the United
States and Canada consisted of a sequence of three
digits. The first digit was an integer between 2 and
9, the second digit was either 0 or 1, and the third
digit was any integer from 1 to 9. How many area
codes were possible? How many area codes starting
with a 4 were possible?

{\bf Solution:}
    \begin{enumerate}
        \item All area codes:\\
        Possible outcomes for 1st digit: 8 \\
        Possible outcomes for 2nd digit: 2 \\
        Possible outcomes for 3rd digit: 9 

        $$ \text{All possible area codes} = 8 \times 2 \times 9  = 144 $$
        \item Area codes starting with a 4:\\
        Possible outcomes for 1st digit: 1 \\
        Possible outcomes for 2nd digit: 2 \\
        Possible outcomes for 3rd digit: 9

        $$ \text{All possible area codes starting with 4} = 1 \times 2 \times 9  = 18 $$
\end{enumerate}
% EXERCISE #3

\item Seven different gifts are to be distributed among
10 children. How many distinct results are possible
if no child is to receive more than one gift?

{\bf Solution:} \\
    Out of 10 children, 7 of them will get 1 gift each and 3 will get no gift. Of the 7 children that receive gifts, there are 7! possible assignments of the gifts. \\
    $$ \text{Possible distinct results} = \begin{pmatrix} 
    10 \\ 7 
    \end{pmatrix}7! = 120 \times 5040 = 604800 $$

% EXERCISE #4
\item Prove that
\[
  \begin{pmatrix}
  n+m \\
  r
  \end{pmatrix}
  =\begin{pmatrix}
  n\\ 0
  \end{pmatrix}\begin{pmatrix}
  m\\ r
  \end{pmatrix}+\begin{pmatrix}
  n\\ 1
  \end{pmatrix}\begin{pmatrix}
  m\\ r-1
  \end{pmatrix}+\cdots +
  \begin{pmatrix}
  n\\ r
  \end{pmatrix}\begin{pmatrix}
  m\\ 0
  \end{pmatrix};
\]
{\bf Solution:}
    \begin{enumerate}
        \item The left hand side is the number of ways to choose $r$ things from $n+m$ things 
        \item $n+m$ can be divided into $n$ number of things and $m$ number of things
        \item Now with one group of $n$ things and one group of $m$ things, choosing $r$ can be done by choosing 0 from the group of $n$ things and $r$ from $m$ things,
        \item Or $1$ thing from $n$ and $r-1$ thing from $m$
        \item Or 2 things from $n$ and $r-2$ things from $m$
        \\ \dots and so on
        \item The sum of the ways that $r$ things can be chosen from a group of $n$ things and a group of $m$ things is the same as the number of ways to choose $r$ from $n+m$ things (the left hand side)
        \item The left hand side and the right hand side are equal
    \end{enumerate}
and then prove 
\[
 \begin{pmatrix}
2n \\
n
\end{pmatrix}
=\sum_{k=0}^{n}
 \begin{pmatrix}
n \\
k
\end{pmatrix}^2.
\]
{\bf Solution:}
    Using the previous proof, let $m=n=r$ 
    $$
    \begin{pmatrix} 
    n+m \\ r 
    \end{pmatrix} =
    \begin{pmatrix}
    n+n \\ n 
    \end{pmatrix} = 
    \begin{pmatrix}
    2n \\ n 
    \end{pmatrix} =
    \begin{pmatrix}
    n\\ 0
    \end{pmatrix}
    \begin{pmatrix}
    n\\ n
    \end{pmatrix} +
    \begin{pmatrix}
    n\\ 1
    \end{pmatrix}\begin{pmatrix}
    n\\ n-1
    \end{pmatrix}
    +\cdots +
    \begin{pmatrix}
    n\\ n
    \end{pmatrix}\begin{pmatrix}
    n \\ 0
    \end{pmatrix} $$
    $$= \sum_{k=0}^{n}
        \begin{pmatrix}
        n \\
        k
        \end{pmatrix}
        \begin{pmatrix}
        n \\
        n - k
        \end{pmatrix}$$
    $$\text{Using the fact that}
        \begin{pmatrix}
        n \\
        k
        \end{pmatrix}
        = \frac{n!}{k!(n-k)!}
        = \frac{n!}{(n-(n-k))!(n-k!)}
        = \begin{pmatrix}
        n \\
        n - k
        \end{pmatrix}$$
	$$\text{Then therefore}\sum_{k=0}^{n}
        \begin{pmatrix}
        n \\
        k
        \end{pmatrix}
        \begin{pmatrix}
        n \\
        n - k
        \end{pmatrix}
        =\sum_{k=0}^{n}
        \begin{pmatrix}
        n \\
        k
        \end{pmatrix}
        \begin{pmatrix}
        n \\
        k
        \end{pmatrix}
        =\sum_{k=0}^{n}
        \begin{pmatrix}
        n \\
        k
        \end{pmatrix}^2
    $$
\end{enumerate}

%\rule{\textwidth}{0.4pt}
\noindent


\end{document}​