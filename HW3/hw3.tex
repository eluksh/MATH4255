% Homework 3
% Emily Lukshin
% MATH 4255 Spring 2020
% February 26 2020

% Preamble

\documentclass{report}
\usepackage[left=1in,right=1in,top=1in,bottom=1in]{geometry}
\setlength{\headheight}{23pt}
\usepackage{amsthm}
\usepackage{amsfonts}
\usepackage{amsmath}
\usepackage{amssymb}
\usepackage{mathrsfs}
\usepackage{multirow}
\usepackage{verbatim}
\usepackage{url}
\usepackage{yfonts}
\usepackage{fancyhdr}
\pagestyle{fancy}
\usepackage[colorlinks]{hyperref}
\usepackage{wrapfig}
\usepackage{graphicx}
\begin{document}

% Header

\fancyhead{}
\fancyfoot{}
\lhead{\large \textbf{Math 4255/5255}}
\chead{\Large \textbf{Homework 3}}
\rhead{\large \textbf{Emily Lukshin}}

% Custom Commands

\newcommand{\N}{\mathbb{N}}
\newcommand{\Z}{\mathbb{Z}}
\newcommand{\Q}{\mathbb{Q}}
\newcommand{\R}{\mathbb{R}}
\newcommand{\C}{\mathbb{C}}
\newcommand{\D}{\mathbb{D}}
\newcommand{\<}{\left\langle}
\renewcommand{\>}{\right\rangle}
\renewcommand{\Re}[1]{\text{Re}\ #1}
\renewcommand{\Im}[1]{\text{Im}\ #1}
\renewcommand{\mod}[1]{(\operatorname{mod}#1)}

% Document
\noindent
{\bf Please submit your solutions (both tex and PDF files) to WyoCourses; and hand in a printed version in class. If you are a graduate student, please also finish Problem 6 and 7.}

\medskip


The following problems are mainly selected from Ross's book. 

\begin{enumerate}
% Exercise 1
\item Ninety-eight percent of all babies survive delivery.
However, 15 percent of all births involve Cesarean
(C) sections, and when a C section is performed,
the baby survives 96 percent of the time. If a randomly
chosen pregnant woman does not have a
C section, what is the probability that her baby
survives?

{\bf Solution:} Let $C$ be the event of a C section and $S$ be the event of survival. From the information given, we know $P(S) = .98$, $P(C)=.15$, $P(S\vert C)=.96$, $P(C^c)=1-P(C)=1-.15=.85$ and we are trying to find  $P(S \vert C^c)$.
\begin{align*}
P(S\vert C^c) &= \frac{P(SC^c)}{P(C^c)} \\
&= \frac{P(S)-P(SC)}{P(C^c)} \\
&= \frac{P(S) - P(S\vert C)P(C)}{P(C^c)} \\
&=\frac{0.98-(0.96)(0.15)}{.85} \\
&=\frac{.836}{.85} \\
&\approx .9835
\end{align*}

% Exercise 2
\item A total of 46 percent of the voters in a certain city
classify themselves as Independents, whereas 30
percent classify themselves as Liberals and 24 percent
say that they are Conservatives. In a recent
local election, 35 percent of the Independents, 62
percent of the Liberals, and 58 percent of the Conservatives
voted. A voter is chosen at random.

Given that this person voted in the local election,
what is the probability that he or she is
\begin{enumerate}
\item an Independent?
\item a Liberal?
\item a Conservative?
\item What fraction of voters participated in the
local election?
\end{enumerate}

{\bf Solution:}
Let $V$ be the event that a voter voted in the local election, $I$ be the event that a voter is Independent, $L$ be the event that a voter is Liberal, and $C$ be the event that a voter is Conservative. From the information given, we know $P(I) = .46$, $P(L) = .3$, $P(C)=.24$, and $P(V\vert I) = .35$, $ P(V\vert L) = .62$, $P(V\vert C)=.58$. The problem is asking (a) $P(I\vert V)=?$, (b) $P(L\vert V)=?$, (c)$P(C\vert V)=?$, and (d) $P(V)=?$. Because this information will be needed to solve the other parts, I will start by solving (d).\smallbreak
\setlength{\parindent}{5pt}(d)
\setlength{\parindent}{0pt}
\begin{align*}
    P(V) &= P(V\vert I)P(I)+P(V\vert L)P(L)+P(V\vert C)P(C) \\
    &= (.35)(.46)+(.62)(.3)+(.58)(.24) \\
    &= .161 + .186 + .1392 \\
    &= .4862
\end{align*}
\begin{enumerate}
    \item 
        \begin{align*}
            P(I\vert V) &= \frac{P(IV)}{P(V)} \\
            &= \frac{P(V\vert I)P(I)}{P(V)} \\
            &= \frac{(.35)(.46)}{.4862} \\
            &= \frac{.161}{.4862} \\
            &\approx .3311
        \end{align*}
    \item
        \begin{align*}
            P(L\vert V) &= \frac{P(LV)}{P(V)} \\
            &= \frac{P(V\vert L)P(L)}{P(V)} \\
            &= \frac{(.62)(.3)}{.4862} \\
            &= \frac{.186}{.4862} \\
            &\approx .3826
        \end{align*}
    \item 
        \begin{align*}
            P(C\vert V) &= \frac{P(CV)}{P(V)} \\
            &= \frac{P(V\vert C)P(C)}{P(V)} \\
            &= \frac{(.58)(.24)}{.4862} \\
            &= \frac{.1392}{.4862} \\
            &\approx .2863
        \end{align*}
\end{enumerate} 
\smallskip

% Exercise 3
\item A total of 48 percent of the women and 37 percent
of the men that took a certain “quit smoking” class
remained nonsmokers for at least one year after
completing the class. These people then attended
a success party at the end of a year. If 62 percent
of the original class was male, what percentage of the original class attended
the party?

{\bf Solution:} Let $S$ be the event that a person in the class attends the success party, $M$ be the event that a person in the class is male, and $F$ be the event that a person in the class is female. From the information given, we know $P(S\vert F)=.48$, $P(S\vert M)=.37$, $P(M)=.62$ and we are trying to find $P(S) = ?$. Because $M$ and $F$ are mutually exclusive, we also know $P(F)=1-P(M)=1-.62=.38$.
\begin{align*}
    P(S) &= P(S\vert M)P(M)+P(S\vert F)P(F) \\
    &=(.37)(.62)+(.48)(.38) \\
    &=.2294+.1824 \\
    &=.4118
\end{align*}


% Exercise 4
\item With probability .6, the present was hidden by
mom; with probability .4, it was hidden by dad.
When mom hides the present, she hides it upstairs
70 percent of the time and downstairs 30 percent
of the time. Dad is equally likely to hide it upstairs
or downstairs.
\begin{enumerate}
\item What is the probability that the present is
upstairs?
\item Given that it is downstairs, what is the probability
it was hidden by dad?
\end{enumerate}

{\bf Solution:} Let $M$ be the event that mom hid the present, $D$ be the event that dad hid the present, $U$ be the event that the present is hidden upstairs, and $N$ be the event that present is hidden downstairs. From the information given, we know $P(M)=.6$, $P(D)=.4$, $P(U\vert M)=.7$, $P(N\vert M)=.3$, $P(U\vert D) = .5$, and $P(N\vert D) =.5$. The problem is asking to find (a) $P(U)=?$ and (b) $P(D\vert N)=?$.
\begin{enumerate}
    \item 
        \begin{align*}
            P(U) &= P(U\vert M)P(M)+P(U\vert D)P(D) \\
            &= (.7)(.6)+(.5)(.4) \\
            &= .42+.2 \\
            &= .62
        \end{align*}
    \item Using the value we found for $P(U)$ in (a) and knowing that $U$ and $N$ are mutually exclusive, we know $P(N)=1-P(U)=1-.62=.38$
        \begin{align*}
            P(D\vert N) &= \frac{P(DN)}{P(N)} \\
            &=\frac{P(N\vert D) P(D)}{P(N)} \\
            &=\frac{(.5)(.4)}{.38} \\
            &= \frac{.2}{.38} \\
            &\approx .5263
        \end{align*}
\end{enumerate}

% Exercise 5
\item A simplified model for the movement of the price
of a stock supposes that on each day the stock’s
price either moves up 1 unit with probability p or
moves down 1 unit with probability $1 - p$. The
changes on different days are assumed to be independent.
\begin{enumerate}
\item What is the probability that after 2 days the
stock will be at its original price?
\item What is the probability that after 3 days the
stock’s price will have increased by 1 unit?
\item Given that after 3 days the stock’s price has
increased by 1 unit, what is the probability
that it went up on the first day?
\end{enumerate}

{\bf Solution:} Let $E_1$ be the event that the price increases by 1 unit on day 1, $E_2$ be the event that the price increases by 1 unit on day 2 and $E_3$ be the event that the price increases by 1 unit on day 3. Therefore, $E_1^c$, $E_2^c$ and $E_3^c$ are the events that the price decreases by 1 unit on their respective day. From the information given in the problem, we know $P(E_1)$, $P(E_2)$, $P(E_3)$ are equal to $p$, and $P(E_1^c)$, $P(E_2^c)$, and $P(E_3^c)$ are equal to $(p-1)$.
\begin{enumerate}
    \item For the stock to be at its original price after two days, the price has to  increase on day 1 then decrease on day 2 $(E_1E_2^c)$, OR decrease on day 1 and increase on day 2 $(E_1^cE_2)$. Therefore, the probability is given as
        \begin{align*}
            P(E_1E_2^c\cup E_1^cE_2)&= P(E_1E_2^c)+P(E_1^cE_2)\\
            &=P(E_1)P(E_2^c)+P(E_1^c)P(E_2) \\
            &=p(1-p)+p(1-p) \\
            &=2p(1-p)
        \end{align*}
    \item For the price to increase by 1 unit after 3 days, the price has to increase on both day 1 and 2 then decrease on day 3 $(E_1E_2E_3^c)$, OR increase on day 1 then decrease on day 2 then increase on day 3$(E_1E_2^cE_3)$, OR decrease on day 1 then increase on both day 2 and 3 $(E_1^cE_2E_3)$. Therefore, the probability is given as
        \begin{align*}
            P(E_1E_2E_3^c\cup E_1E_2^cE_3 \cup E_1^cE_2E_3) &= P(E_1E_2E_3^c)+P(E_1E_2^cE_3)+P(E_1^cE_2E_3)\\
            &=P(E_1)P(E_2)P(E_3^c)+P(E_1)P(E_2^c)P(E_3)+P(E_1^c)P(E_2)P(E_3) \\
            &=p\cdot p (p-1)+p\cdot p (p-1)+p\cdot p (p-1) \\ 
            &=3p^2(p-1)
        \end{align*}
    \item Let $U$ be the event that the stock's price has increased by 1 unit after 3 days. From part (b), we know $P(U)=3p^2(1-p)$.
        \begin{align*}
            P(E_1\vert U)&=\frac{P(E_1U)}{P(U)} \\
            &=\frac{P(E_1E_2^cE_3)+P(E_1E_2E_3^c)}{P(U)} \\
            &=\frac{p\cdot p(1-p)+p\cdot p(1-p)}{3p^2(1-p)} \\
            &=\frac{2p^2(1-p)}{3^2(1-p)} \\
            &=\frac{2}{3}
        \end{align*}
\end{enumerate}
	
\end{enumerate}

\end{document}​