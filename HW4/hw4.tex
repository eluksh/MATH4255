% Homework 4
% Emily Lukshin
% MATH 4255 Spring 2020
% March 3 2020

% Preamble 

\documentclass{report}
\usepackage[left=1in,right=1in,top=1in,bottom=1in]{geometry}
\setlength{\headheight}{23pt}
\usepackage{amsthm}
\usepackage{amsfonts}
\usepackage{amsmath}
\usepackage{amssymb}
\usepackage{mathrsfs}
\usepackage{multirow}
\usepackage{verbatim}
\usepackage{url}
\usepackage{yfonts}
\usepackage{fancyhdr}
\pagestyle{fancy}
\usepackage[colorlinks]{hyperref}
\usepackage{wrapfig}
\usepackage{graphicx}
\begin{document}

% Header

\fancyhead{}
\fancyfoot{}
\lhead{\large \textbf{Math 4255/5255}}
\chead{\Large \textbf{Homework 4}}
\rhead{\large \textbf{Emily Lukshin}}

% Custom Commands 

\newcommand{\N}{\mathbb{N}}
\newcommand{\Z}{\mathbb{Z}}
\newcommand{\Q}{\mathbb{Q}}
\newcommand{\R}{\mathbb{R}}
\newcommand{\C}{\mathbb{C}}
\newcommand{\D}{\mathbb{D}}
\newcommand{\<}{\left\langle}
\renewcommand{\>}{\right\rangle}
\renewcommand{\Re}[1]{\text{Re}\ #1}
\renewcommand{\Im}[1]{\text{Im}\ #1}
\renewcommand{\mod}[1]{(\operatorname{mod}#1)}

% Document

\noindent
{\bf Please submit your solutions (both tex and PDF files) to WyoCourses; and hand in a printed version in class. If you are a graduate student, please also finish Problem 6.}

\medskip


The following problems are mainly selected from Ross's book. 

\begin{enumerate}

% EXERCISE 1

\item Let $X$ be the winnings of a gambler. Let $p(i)=P(X=i)$ and suppose that
\[
  p(0)=1/3; p(1)=p(-1)=13/55; p(2)=p(-2)=1/11; p(3)=p(-3)=1/165.
\]
Compute the conditional probability that the gambler wins $i$, $i=1,2,3$, given that he wins a positive amount. 

{\bf Solution:} Let E be the event that the gambler wins a positive amount of money.
\begin{align*}
    P(E) &= P(1)+P(2)+P(3) \\
    &=13/55+1/11+1/165 \\
    &=55/165 \\
    &=1/3 \\
    P(X=i\vert E) &= \frac{p(i)}{P(E)} \\
    &=\frac{p(i)}{\frac{1}{3}} \\
    &=3p(i)
\end{align*}

% EXERCISE 2

\item Four buses carrying $148$ students from the same
school arrive at a football stadium. The buses carry, respectively,
$40, 33, 25$, and $50$ students. One of the students is
randomly selected. Let X denote the number of students
who were on the bus carrying the randomly selected student.
One of the $4$ bus drivers is also randomly selected.
Let Y denote the number of students on her bus.

\begin{enumerate}
	\item Which of $E[X]$ or $E[Y]$ do you think is larger? Why?
	\item Compute $E[X]$ and $E[Y]$. 
\end{enumerate}

{\bf Solution:}
\begin{enumerate}
    \item $E[X] > E[Y]$, because there are 148 students to choose from, but only 4 bus drivers.
    \item 
        \begin{align*}
            E[X]&=40\times 40/148 + 33 \times 33/148 + 25\times 25/148 + 50 \times 50/148 \\
            &= 5814/148 \\
            &\approx 39.28 \\
            E[Y]&=40\times1/4+33\times1/4+25\times1/4+50\times 1/4 \\
            &=148/4 \\
            &=37
        \end{align*}
\end{enumerate}

% EXERCISE 3

\item Prostate cancer is the most common type of cancer found in males.
As an indicator of whether a male has prostate cancer, doctors often perform a
test that measures the level of the prostate-specific antigen (PSA) that is produced
only by the prostate gland. Although PSA levels are indicative of cancer, the test is
notoriously unreliable. Indeed, the probability that a noncancerous man will have
an elevated PSA level is approximately 0:135, increasing to approximately $0:268$ if
the man does have cancer. If, on the basis of other factors, a physician is $70$ percent
certain that a male has prostate cancer, what is the conditional probability that he
has the cancer given that
\begin{enumerate}
	\item the test indicated an elevated PSA level?
	\item the test did not indicate an elevated PSA level?
\end{enumerate}

Denote by $C$ the event that the patient has cancer and by $E$ the event that
the test indicates and elevated PSA level.

{\bf Solution:} From the information given, we know $P(E\vert C^c)=.135$, $P(E\vert C) =.268$, $P(C)=.7$, and $P(C^c)=1-.7=.3$. The problem is asking (a)$P(C\vert E)=?$ and (b)$P(C\vert E^c)=?$
\begin{enumerate}
    \item 
    \begin{align*}
        P(C\vert E) &= \frac{P(CE)}{P(E)} \\
        &=\frac{P(E^c\vert C)P(C)}{P(E\vert C)P(C)+P(E\vert C^c)P(C^c)} \\
        &=\frac{(0.268)(0.7)}{(0.268)(0.7)+(0.135)(0.3)} \\
        &=\frac{0.1876}{0.2281}\\
        &\approx 0.8224
    \end{align*}
    \item
    \begin{align*}
        P(C\vert E^c) &= \frac{P(CE^c)}{P(E^c)} \\
        &=\frac{P(E^c\vert C)P(C)}{P(E^c\vert C)P(C)+P(E^c\vert C^c)P(C^c)} \\
        &=\frac{(1-0.268)(0.7)}{(1-0.268)(0.7)+(0.135)(0.3)} \\
        &=\frac{0.5124}{0.7719} \\
        &\approx 0.6638
    \end{align*}
\end{enumerate}

% EXERCISE 4

\item Approximately $80,000$ marriages took place in the
state of New York last year. Estimate the probability that
for at least one of these couples,
\begin{enumerate}
	\item both partners were born on April 30;
	\item both partners celebrated their birthday on the same
	day of the year.
\end{enumerate}
State your assumptions.

{\bf Solution:}
\begin{enumerate}
    \item We are assuming Poisson distribution to approximate since the probability that both partners were born on April 30, $p=(1/365)(1/365)\approx 7.506\times 10^{-6}$, is very small and $n=80,000$.
    $$ \lambda = np = (80000) (7.506\times10^{-6})\approx 0.6$$
    Let $X$ be a random variable of the number of couples with both partners born on April 30
    \begin{align*}
        P\{ X\geq 1\} &= 1-P\{ X=0\} \\
        &=1-e^{-\lambda} \frac{\lambda^0}{0!}\\
        &=1-e^{-0.6} \\
        &\approx0.4512 
    \end{align*}
   \item We are assuming Poisson distribution to approximate since the probability that both partners have the same birthday, $p=(1/365)\approx 0.0027$, is small and $n=80,000$.
   $$ \lambda = np = (80000)(0.0027)\approx 219.178$$
    Let $X$ be a random variable of the number of couples with both partners born on April 30
    \begin{align*}
        P\{ X\geq 1\} &= 1-P\{ X=0\} \\
        &=1-e^{-\lambda} \frac{\lambda^0}{0!}\\
        &=1-e^{-219.178} \\
        &\approx 1
    \end{align*}
    
\end{enumerate}

% EXERCISE 5

\item Consider n independent trials, each of which results
in one of the outcomes $1, \cdots, k$ with respective probabilities
$p_1,\cdots, p_k$, 
$\sum_{i=1}^{k}p_k=1$. Show that if all the $p_i$ are
small, then the probability that no trial outcome occurs
more than once is approximately equal to $\exp(-n(n-1)\sum_i p_i^2/2)$. 

{\bf Solution:} The probability, $p$, that 2 of the trial outcomes will be the same is
$$p=\sum_{i=1}^{k}p_i^2$$
Of n trials, the possibility that 2 of the outcomes are the same is
$${n \choose 2} =\frac{n(n-1)}{2}$$
Using Poisson distribution, 
$$\lambda = np = \frac{n(n-1)}{2}\sum_{i=1}^{k}p_i^2$$ 
Let $X$ be the random variable of the number of 2 trials with the same outcome. The probability that no 2 outcomes are the same is
\begin{align*}
    P\{ X=0\} &= e^{-\lambda}\frac{\lambda^0}{0!} \\
    &=e^{-\lambda} \\
    &=exp \left ( \frac{n(n-1)}{2}\sum_{i=1}^{k}p_i^2 \right)
\end{align*}

	
\end{enumerate}

\end{document}​